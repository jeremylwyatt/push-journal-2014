\section{Introduction}\label{sec:Introduction}

Predicting what will happen next is central to intelligent behaviour \cite{craik1967nature}. If an agent cannot predict the effects of its actions it cannot autonomously plan a course of action to achieve its goals. Modelling a robot's interactions with the world, so that it can make useful predictions about the effects of its actions, is a challenging set of problems. This paper is about how to predict the motions of a rigid object when manipulated by a robot. While simple to pose, this problem is not easy to solve. This paper argues that predicting future behaviour on the basis of past experience is unavoidably a problem of learning. It also shows such predictors can be learned tabula rasa, and how the learned models can be transferred to new objects and actions. The paper focuses on kinematic models of object behaviour, since forces and masses are assumed to be unknown at prediction time. Within this scope the paper makes the following contributions. First it presents a {\em modular machine learning} solution to object motion prediction problems, and shows for the first time that this outperforms physics simulation on a variety of real objects.  Second, it shows for the first time results for transfer learning on real objects. It thus extends our previous work \cite{kopicki_prediction_2010,kopicki-etal-icra11}  where an initial non-modular scheme was tested in simulation, and on a single real object.  This paper tests three hypotheses with respect to real objects. Hypothesis~1 is that a {\em modular learning} approach can outperform physics engines for prediction of rigid body motion.  Hypothesis~2 is that by factorising these modular predictors they can be transferred to make predictions about novel actions. Hypothesis~3 is that by factorising learning can be transferred  to make predictions about novel shapes.

The paper is structured as follows.  Section~\ref{sec:schema} describes the problem informally and the benefits of a modular learning approach to solving it. Next, Section~\ref{sec:Representations} introduces representations of object motion, enabling a formal problem statement in Section~\ref{sec:PredictionProblem} and formulation in both regression and density estimation frameworks. Section~\ref{sec:InfoForPrediction} incorporates contact information, and Section~\ref{sec:Factors} describes how we can factor the learner by the contacts to achieve transfer learning. Section~\ref{sec:Implementation} gives implementation details, Section~\ref{sec:Experiment} the experimental method, and Section~\ref{sec:Results}
the corresponding results. Section~\ref{sec:Background} reviews related work.  We finish with a discussion in Section~\ref{sec:Discussion}.